\subsection{Math Functions}
\texttt{{\large\textbf{ceil}}}\textsf{(real:x):real}\\
 Round to the smallest integral value not less than x.

\hrulefill

\texttt{{\large\textbf{round}}}\textsf{(real:x):real}\\
 Round to the nearest integral value.

\hrulefill

\texttt{{\large\textbf{floor}}}\textsf{(real:x):real}\\
 Round to the largest integral value not greater than x.

\hrulefill

\texttt{{\large\textbf{sqrt}}}\textsf{(real:x):real}\\
 Returns the square root of a number.

\hrulefill

\texttt{{\large\textbf{pow}}}\textsf{(real:x, real:y):real}\\
 Returns a number x raised to the power y.

\hrulefill

\texttt{{\large\textbf{exp}}}\textsf{(real:x):real}\\
 Returns the base-e exponential of x.

\hrulefill

\texttt{{\large\textbf{log}}}\textsf{(real:x):real}\\
 Returns the base-e logarithm of x.

\hrulefill

\texttt{{\large\textbf{log10}}}\textsf{(real:x):real}\\
 Returns the base-10 logarithm of x.

\hrulefill

\texttt{{\large\textbf{pi}}}\textsf{(void):real}\\
 Returns the value of PI.

\hrulefill

\texttt{{\large\textbf{sgn}}}\textsf{(real:x):real}\\
 Returns 1 if the argument is greater than zero, 0 if argument is 0, otherwise -1.

\hrulefill

\texttt{{\large\textbf{abs}}}\textsf{(real:x):real}\\
 Returns the absolute value of a number.

\hrulefill

\texttt{{\large\textbf{sin}}}\textsf{(real:x):real}\\
 Computes the sine of x (radians)

\hrulefill

\texttt{{\large\textbf{cos}}}\textsf{(real:x):real}\\
 Computes the cosine of x (radians)

\hrulefill

\texttt{{\large\textbf{tan}}}\textsf{(real:x):real}\\
 Computes the tangent of x (radians)

\hrulefill

\texttt{{\large\textbf{asin}}}\textsf{(real:x):real}\\
 Computes the arc sine of x, result is in radians, -pi/2 to pi/2.

\hrulefill

\texttt{{\large\textbf{acos}}}\textsf{(real:x):real}\\
 Computes the arc cosine of x, result is in radians, 0 to pi.

\hrulefill

\texttt{{\large\textbf{atan}}}\textsf{(real:x):real}\\
 Computes the arc tangent of x, result is in radians, -pi/2 to pi/2.

\hrulefill

\texttt{{\large\textbf{atan2}}}\textsf{(real:x, real:y):real}\\
 Computes the arc tangent using both x and y to determine the quadrant of the result, result is in radians.

\hrulefill

\texttt{{\large\textbf{sind}}}\textsf{(real:x):real}\\
 Computes the sine of x (degrees)

\hrulefill

\texttt{{\large\textbf{cosd}}}\textsf{(real:x):real}\\
 Computes the cosine of x (degrees)

\hrulefill

\texttt{{\large\textbf{tand}}}\textsf{(real:x):real}\\
 Computes the tangent of x (degrees)

\hrulefill

\texttt{{\large\textbf{asind}}}\textsf{(real:x):real}\\
 Computes the arc sine of x, result is in degrees, -90 to 90.

\hrulefill

\texttt{{\large\textbf{acosd}}}\textsf{(real:x):real}\\
 Computes the arc cosine of x, result is in degrees, 0 to 180.

\hrulefill

\texttt{{\large\textbf{atand}}}\textsf{(real:x):real}\\
 Computes the arc tangent of x, result is in degrees, -90 to 90.

\hrulefill

\texttt{{\large\textbf{atan2d}}}\textsf{(real:x, real:y):real}\\
 Computes the arc tangent using both x and y to determine the quadrant of the result, result is in degrees.

\hrulefill

\texttt{{\large\textbf{nan}}}\textsf{(void):real}\\
 Returns the non-a-number (NAN) value.

\hrulefill

\texttt{{\large\textbf{isnan}}}\textsf{(number):boolean}\\
 Returns true if the argument is NaN.

\hrulefill

\texttt{{\large\textbf{mod}}}\textsf{(integer:x, integer:y):integer}\\
 Returns the remainder after integer division of x by y.

\hrulefill

\texttt{{\large\textbf{sum}}}\textsf{(...):real}\\
 Returns the numeric sum of all values passed to the function. Arguments can be arrays or numbers.

\hrulefill

\texttt{{\large\textbf{min}}}\textsf{(...):real}\\
 Returns the minimum of the numeric arguments.

\texttt{{\large\textbf{min}}}\textsf{(array):real}\\
 Returns the minimum value in an array of numbers

\emph{Notes:} Returns the minimum numeric value.

\hrulefill

\texttt{{\large\textbf{max}}}\textsf{(...):real}\\
 Returns the maximum of the passed numeric arguments.

\texttt{{\large\textbf{max}}}\textsf{(array):real}\\
 Returns the maximum value in an array of numbers

\emph{Notes:} Returns the maximum numeric value.

\hrulefill

\texttt{{\large\textbf{mean}}}\textsf{(...):real}\\
 Returns the mean (average) value all values passed to the function. Arguments can be arrays or numbers.

\hrulefill

\texttt{{\large\textbf{stddev}}}\textsf{(...):real}\\
 Returns the sample standard deviation of all values passed to the function. Uses Bessel's correction (N-1). Arguments can be arrays or numbers.

\hrulefill

\texttt{{\large\textbf{gammaln}}}\textsf{(real):real}\\
 Computes the logarithm of the Gamma function.

\hrulefill

\texttt{{\large\textbf{pearson}}}\textsf{(array:x, array:y):real}\\
 Calculate the Pearson linear rank correlation coefficient of two arrays.

\hrulefill

\texttt{{\large\textbf{besj0}}}\textsf{(real:x):real}\\
 Computes the value of the Bessel function of the first kind, order 0, J0(x)

\hrulefill

\texttt{{\large\textbf{besj1}}}\textsf{(real:x):real}\\
 Computes the value of the Bessel function of the first kind, order 1, J1(x)

\hrulefill

\texttt{{\large\textbf{besy0}}}\textsf{(real:x):real}\\
 Computes the value of the Bessel function of the second kind, order 0, Y0(x)

\hrulefill

\texttt{{\large\textbf{besy1}}}\textsf{(real:x):real}\\
 Computes the value of the Bessel function of the second kind, order 1, Y1(x)

\hrulefill

\texttt{{\large\textbf{besi0}}}\textsf{(real:x):real}\\
 Computes the value of the modified Bessel function of the first kind, order 0, I0(x)

\hrulefill

\texttt{{\large\textbf{besi1}}}\textsf{(real:x):real}\\
 Computes the value of the modified Bessel function of the first kind, order 1, I1(x)

\hrulefill

\texttt{{\large\textbf{besk0}}}\textsf{(real:x):real}\\
 Computes the value of the modified Bessel function of the second kind, order 0, K0(x)

\hrulefill

\texttt{{\large\textbf{besk1}}}\textsf{(real:x):real}\\
 Computes the value of the modified Bessel function of the second kind, order 1, K1(x)

\hrulefill

\texttt{{\large\textbf{erf}}}\textsf{(real):real}\\
 Calculates the value of the error function

\hrulefill

\texttt{{\large\textbf{erfc}}}\textsf{(real):real}\\
 Calculates the value of the complementary error function
